\documentclass[twoside,a4paper,11pt]{article}

\usepackage{exscale}
\usepackage{latexsym}
\usepackage{docu}
\usepackage{a4wide}

\usepackage{times}
\renewcommand{\ttdefault}{cmtt}

\usepackage{hyperref}

\hypersetup{colorlinks=false,
   pdfborder=0 0 0,
   pdftitle=Preparing Slide Presentations with the PPTeX Package,
   pdfauthor=Ulrich Bodenhofer}


\def\AMS{{\protect\usefont{OMS}{cmsy}{m}{n}%
A\kern-.1667em\lower.5ex\hbox{M}\kern-.125emS}}

\def\PPTeX{{\upshape PP\TeX{}}}


\title{\bfseries\sffamily\LARGE Preparing Slide Presentations with the\\ \PPTeX\ Package}
\author{\sffamily\Large Ulrich Bodenhofer ({\em
ulrich@bodenhofer.com})}
\date{}

\begin{document}
\maketitle

\vspace*{1cm}

\tableofcontents
\clearpage


\section{Introduction}

In 2000, the author developed a \LaTeX\ document class for his
former employer, Software Competence Center Hagenberg, with the aim
to emulate the corporate design of this institution's MS PowerPoint
template. The given package---{\tt pptex.cls}---is a new version and
generalization of this package (that was called {\tt
scch-slides.cls} then). The main difference between the former class
{\tt scch-slides.cls} and {\tt pptex.cls} is that different layouts,
including the SCCH layout, are possible. The set of available
layouts is extensible. Moreover, {\tt pptex.cls} now supports a 4:3
page/slide format of 10$\times$7.5in by default (the same as MS
PowerPoint), whereas {\tt scch-slides.cls} was only able to use A4
format.

The PP\TeX\ slide document class  can be used in the following way:
\begin{quote}
|\documentclass|\oarg{options}|{pptex}|
\end{quote}

The following options are available:
\begin{description}
\item[\texttt{display}] enables animation features for beamer
presentations
\item[\texttt{printout}] turns off all animation features (recommended for
handouts, printed slides, etc.)
\item[\texttt{startblank}] adds a blank black slide as first page
\item[\texttt{endblank}] adds a blank black slide as last page
\item[\texttt{a4paper}] switches back to A4 paper (only for
backwards compatibility; not recommended)
\end{description}

Moreover, the name of a layout has to be passed as an option as
well. If no layout is specified, \LaTeX\ displays an error message.

The whole class is designed for use\NOTE{} with PDF\LaTeX\ only. Any
attempt to process a \PPTeX\ document with ordinary \LaTeX\ results
in an error message.

\section{Layouts}

As mentioned above, a layout has to be specified as option in the
\verb+\documentclass+ command. Presently, three layouts are
available:
\begin{description}
\item[\texttt{fhooe}] layout resembling the corporate design of
University of Applied Sciences Upper Austria (Fachhochschule Ober\"osterreich --- FH O\"O),
with international logo
\item[\texttt{htll}] layout resembling the corporate design of
HTBLA Leonding
\item[\texttt{ub}] clean steelblue layout used by author for some time (similar to uda, but without logos)
\item[\texttt{jku}] layout resembling the new corporate design of
Johannes Kepler University Linz (JKU, 2015)
\item[\texttt{bioinf}] layout in the corporate design of the
Institute of Bioinformatics, Johannes Kepler University Linz
\item[\texttt{jkuold}] layout resembling the old corporate design of
Johannes Kepler University Linz (JKU, around 2003)
\item[\texttt{scch}] layout corresponding to the MS PowerPoint
template of Software Competence Center Hagenberg (SCCH, as of 2006)
\item[\texttt{scchh}] the same as \texttt{scchh}, but without using
the Verdana font
\item[\texttt{scchold}] obsolete layout corresponding to the previous MS PowerPoint
template of Software Competence Center Hagenberg (SCCH)
\item[\texttt{uda}] layout for presentations related to Austrian Partner of International Universities (UDA)
\item[\texttt{udaold}] obsolete layout for presentations related to previous Unversity of Derby in Austria (UDA)
\end{description}
The layouts \verb+jkuold+, \verb+scchold+ and \verb+udaold+ are
available for 4:3 format (default) and A4 format (when the
\verb+a4paper+ option is being used) in order to provide a
compatibility mode for older presentations. The layouts
\verb+jku+, \verb+bioinf+, \verb+scch+, \verb+scchh+  and \verb+uda+ are newer
ones and have been developed when \PPTeX\ was already available, so
backwards compatibility is not an issue for these layouts.

It is possible to define new layouts, but it is not yet documented
how to do this. If you are interested in writing a new layout,
please contact the author of the package.

\section{Basic Usage}

The \verb+pptex+ class is based on the document class \verb+seminar+
by Timothy Van Zandt and makes use of the \TeX Power bundle written
by Stephan Lehmke.

Similar to the classes \verb+seminar+ and \verb+slides+, slide text
must be enclosed in the \verb+slide+ environment. Internally, the
\verb+pptex+ class works with \verb+picture+ environments. In the
following, we will discuss the convenience functions that have been
implemented in order to make work with these picture environments
easier.

\begin{decl}
|\titleslide|\oarg{author}\arg{title}\arg{subtitle}
\end{decl}
\noindent Produces a title slide, where the meaning of the arguments
should be self-explanatory. Please note that the optional argument
is not used by all layouts (cf.~\ref{ssec:titleslide}).

\begin{decl}
|\simpletextslide|\arg{headline}\arg{text}
\end{decl}
\noindent Produces a slide with one single large text box. The
second argument may contain any valid \LaTeX\ text (including
mathematical formulas, the environments \verb+itemize+,
\verb+enumerate+, \verb+description+, \verb+list+, \verb+verse+,
\verb+quote+, \verb+quotation+, \verb+verbatim+, \verb+center+,
\verb+flushleft+, \verb+flushright+, and many more).

Here is a simple example how to use \verb+\titleslide+ and
\verb+\simpletextslide+:
\begin{quote}
\begin{verbatim}
\documentclass[display,jku]{pptex}

\setfootline{Ulrich Bodenhofer}

\begin{document}

\begin{slide}
\titleslide{PP\TeX}{% title
A Short Introduction} % subtitle
\end{slide}

\begin{slide}
\simpletextslide{Overview}{%
\begin{itemize}
\item Motivation
\item Basic usage
\item Using layouts
\item Miscellaneous issues
\end{itemize}}
\end{slide}

\end{document}
\end{verbatim}
\end{quote}

\begin{decl}
|\simpleslide|\arg{headline}\arg{body}
\end{decl}
\noindent Produces an empty slide. The argument \textarg{body} may
only contain \verb+\put+ commands, since the surrounding environment
is a \verb+picture+ environment. However, the \verb+pptex+ class
contains several convenience macros for several often used
structures which will be described next.

\subsection{Convenience Macros for Positioning Text and Pictures
Freely}

The following macros may only be used inside the \textarg{body}
argument of the \verb+\simpleslide+ command.

\begin{decl}
|\textbox|\oarg{align}\arg{xpos,ypos}\arg{w}\arg{text}
\end{decl}
\noindent This command puts a text box of width \textarg{w} with
content \textarg{text} at position (\textarg{xpos,ypos}).
Normally, the text box is placed such that its left upper corner
is at position (\textarg{xpos,ypos}). This can be overridden with
the optional argument \textoarg{align}, the effect of which is
the same as of the optional argument of the \verb+\makebox+
command inside \verb+picture+ environments:
\begin{description}
\item[{\tt lt} (default):] upper left corner at (\textarg{xpos,ypos})
\item[{\tt l}:] middle of left edge at (\textarg{xpos,ypos})
\item[{\tt lb}:] lower left corner at (\textarg{xpos,ypos})
\item[{\tt rt}:] upper right corner at (\textarg{xpos,ypos})
\item[{\tt r}:] middle of right edge at (\textarg{xpos,ypos})
\item[{\tt rb}:] lower right corner at (\textarg{xpos,ypos})
\item[{\tt t}:] middle of top edge at (\textarg{xpos,ypos})
\item[{\tt b}:] middle of bottom edge at (\textarg{xpos,ypos})
\end{description}
Note that the coordinates (\textarg{xpos,ypos}) need to be scalars
(no units!). The internal unit of measure is cm relative to the
lower left corner of the slide (total size 10$\times$7.5 in $=$
25.4$\times$19.05 cm; 29.7$\times$21cm if the option \verb+a4paper+
is used); analogous for the width \textarg{w}. Concerning the
argument \textarg{text}, the same rules as for the second argument
of the \verb+\simpletextslide+ command apply (see above).

\begin{decl}
|\itembox|\oarg{align}\arg{xpos,ypos}\arg{w}\arg{items} \\
|\enumbox|\oarg{align}\arg{xpos,ypos}\arg{w}\arg{items}
\end{decl}
\noindent These two convenience functions are basically the same as
\verb+\textbox+ except that they already contain \verb+itemize+ and
\verb+enumerate+ environments, respectively.

\begin{decl}
|\clearbox|\oarg{align}\arg{xpos,ypos}\arg{content}
\end{decl}
\noindent Unlike the above three macros, this command allows to
create a simple box without any internal pre-assumption about its
content. This command is mainly useful for inserting graphics.

Examples for most of these macros can be found in the enclosed
demo file \verb+slidedemo.tex+.

\subsection{Further Convenience Macros for Positioning Text and Pictures}

Though the macros in the previous subsection are very flexible, they are
not very user-friendly. In order to provide the user with some simpler, but
more flexible convenience functions, the layouts \verb+jku+ and \verb+bioinf+
provide the following macro:
\begin{decl}
|\defaulttextbox|\arg{text}
\end{decl}
This macro, in conjunction with \verb+\simpleslide+, is equivalent to the
functionality of \verb+\simpletextslide+. However, now the entire box can be
animated and complemented by additional boxes (e.g. with additional text or
figures).

For two-column layouts, the following two macros are available:
\begin{decl}
|\defaulttextboxleft|\arg{text} \\
|\defaulttextboxright|\arg{text}
\end{decl}

\subsection{The Foot Line}

\begin{decl}
|\setfootline|\arg{footline}
\end{decl}
\noindent Determines the content of foot line.
This command can be supplied in the preamble of the document or
inside the document whenever a change of the content of the foot
line is desired.

\subsection{Lengths}\label{ssec:lengths}

Since the \verb+seminar+ class works with magnifications, an
alternative model of lengths is used internally. Therefore,
unfortunately, the usual font-independent \LaTeX\ lengths (units
\verb+cm+, \verb+mm+, and \verb+in+) cannot be used in the normal
way.\NOTE{} Instead, the macros \verb+\semcm+ and \verb+\semin+
need to be used in order to obtain the desired results.
Fortunately, no restrictions apply to the font-dependent units
\verb+em+ and \verb+ex+.

\subsection{Graphics Inclusion}

Graphics inclusion works as usual. However, the user should take
the following two points into account:
\begin{enumerate}
\item Since the \verb+scch-slides+ class is designed for use with
PDF\LaTeX, only JPEG, PDF, and PNG files can be included, but no
encapsulated PostScript (EPS) files. EPS files must converted to
PDF first. This can be done by the GhostScript-based tool
\verb+epstopdf+ which is available on the SCCH common software
server.
\item As described in \ref{ssec:lengths}, font-independent lengths
cannot be used in a straightforward way. Use the macros
\verb+\semcm+ and \verb+\semin+ instead!
\end{enumerate}
It is recommended to place pictures inside \verb+\clearbox+
environments, although the other box commands and
\verb+\simpletextslide+ also work. A simple example:
\begin{quote}
\begin{verbatim}
\clearbox{3,14}{\includegraphics[width=7\semcm]{dummy.jpg}}
\end{verbatim}
\end{quote}

\subsection{Animation}

The \verb+pptex+ class makes use of the \TeX Power bundle's
animation features. Consequently, the reader is referred to the
enclosed manual for more details. Be aware, however, that the
\verb+pptex+ class makes excessive use of the \verb+picture+
environment. Therefore, the \verb+\pause+ command does\NOTE{} not
work. Instead, the constructs \verb+\stepwise+, \verb+\step+, and
all the related macros must be used in order to make animation work
properly. More information is available in the enclosed \TeX Power
manual and the example files.

\subsection{Hyperlinks}

The package \verb+hyperref+ is included by default. Therefore, all
its features can be used in \PPTeX\ slide presentations without any
restrictions. See the \verb+hyperref+ manual for more details.

\subsection{Packages}

The following packages are included by \verb+pptex.cls+ and need not
be included in the preamble of the document:
\begin{enumerate}
\item \verb+color.sty+
\item \verb+graphicx.sty+
\item \verb+hyperref.sty+
\item \verb+url.sty+
\item \verb+texpower.sty+
\end{enumerate}

The \verb+pptex+ document class should work properly with most
standard packages, in particular those belonging to the \AMS\LaTeX\
bundle.

\section{Layout-Dependent Behavior}

While \PPTeX\ tries to maintain layout independence where possible,
there are some functions/issues that depend on the layout that is
actually being used.

\subsection{Optional argument of \texttt{titleslide}
command}\label{ssec:titleslide}

In the layouts \verb+jku+, \verb+bioinf+, \verb+uda+ and
\verb+udaold+, the \verb+titleslide+ command ignores the optional
argument,\NOTE{} whereas the \verb+scch+ layout uses it to typeset
the typical four-line author box to the right of the logo.

\subsection{Page formats}

Although the page/slide size is the same independent of the layout,
the actually available space for text and drawings may be different
for different layouts. The reason is simple that logos and graphical
elements occupy differently much space. So it is almost sure that a
\PPTeX\ document that has been written using the above documented
commands compiles well with different layouts.\NOTE{} It is likely,
however, that adaptations of positions of text and graphics or
adaptations of font sizes are necessary if the layout is changed.

\subsection{Additional fonts}

The layout \verb+jku+ requires the font Arial Black and the layout
\verb+scch+ requires Verdana to work correctly. The 

\subsection{Colors}

Different layouts may define different colors. These colors can be
used as usual with the commands \verb+\color+, \verb+\textcolor+,
\verb+\colorbox+, and \verb+\fcolorbox+ (see documentation on the
\verb+color.sty+ package for more information). The layouts may also
define different convenience macros for using the colors.



\subsubsection{Layout \texttt{fhooe}}

Internally, the colors \verb+fhred+, \verb+fhlre+ (light red), \verb+fhdg+ (dark gray),
\verb+fhlg+ (light gray), \verb+fhvlg+ (very light gray), \verb+fhmg+ (medium gray), \verb+fhwh+ (white),
and \verb+fhbl+ (black) are pre-defined.

The following commands can be used to switch to the color to red for highlighting.
They can be used in the same way as \verb+\bf+:
\begin{decl}
|\fhred| \\
|\red|
\end{decl}

\subsubsection{Layout \texttt{htll}}

Internally, the colors \verb+htlred+, \verb+htldg+ (dark gray),
\verb+htllg+ (light gray), \verb+htlwh+ (white),
and \verb+htlbl+ (black) are pre-defined.

The following commands can be used to switch to the color to red for highlighting.
They can be used in the same way as \verb+\bf+:
\begin{decl}
|\htlred| \\
|\red|
\end{decl}

\subsubsection{Layout \texttt{ub}}

The colors \verb+ubaz+ (dark blue), \verb+ubgr+ (medium gray),
\verb+ubbg+ (very light steel blue), and \verb+ubbl+ (black) are
pre-defined.

\begin{decl}
|\blue|\\
|\grey|
\end{decl}
As above, these commands can be used to switch to the colors for
highlighting. It can be used in the same way as \verb+\bf+.


\subsubsection{Layout \texttt{jku}}

Internally, the colors \verb+jkublue+, \verb+jkupurple+, \verb+jkured+,
\verb+jkuyellow+, \verb+jkugreen+, and \verb+jkulightblue+ are pre-defined
according to the new JKU Corporate Design (2015).

The following commands can be used to switch to the colors for highlighting.
They can be used in the same way as \verb+\bf+:
\begin{decl}
|\blue| \\
|\purple| \\
|\red| \\
|\yellow| \\
|\green| \\
|\lightblue|
\end{decl}

\subsubsection{Layout \texttt{bioinf}}

Internally, the colors \verb+bioaz+ (dark blue of the square in the
logo), \verb+biove+ (dark green, like the darker ends of the DNA
strands in the logo), \verb+bioli+ (light green, like the lighter
ends of the DNA strands in the logo), \verb+biowh+ (white), and
\verb+biobl+ (black) are pre-defined.

\begin{decl}
|\green| \\
|\blue|
\end{decl}
These commands can be used to switch to the colors for highlighting.
They can be used in the same way as \verb+\bf+.

\subsubsection{Layout \texttt{jkuold}}

Internally, the colors \verb+jkure+ (ruby red), \verb+jkugr+ (medium
gray), \verb+jkulg+ (light gray), \verb+jkuwh+ (white), and
\verb+jkubl+ (black) are pre-defined.

\begin{decl}
|\red|
\end{decl}
This command can be used to switch to the colors for highlighting.
It can be used in the same way as \verb+\bf+.

\subsubsection{Layout \texttt{scch}}

Internally, the colors \verb+scchre+ (ruby red), \verb+scchlg+
(light gray), \verb+scchdg+ (dark gray), \verb+scchwh+ (white), and
\verb+scchbl+ (black) are pre-defined, corresponding to the standard
colors of the new SCCH corporate identity.

\begin{decl}
|\red|
\end{decl}
This command can be used to switch to the color that is standard for
highlighting in the corresponding PowerPoint template. It can be
used like a font switch (e.g. \verb+\bf+).

\begin{decl}
|\scchcopyright|
\end{decl}
This command prints the SCCH copyright notice ``\copyright\ Software
Competence Center Hagenberg GmbH''. Note that this is also the
default foot line.

The standard SCCH PowerPoint layout uses Verdana\NOTE{} as default
font for title and subtitle on title pages and for slide headings.
The SCCH layout cannot be used if this font is not available to
PDF\LaTeX. For instructions how to make Verdana usable for
PDF\LaTeX, see Appendix \ref{app:verdana}.

\subsubsection{Layout \texttt{scchh}}

Works in the same way as the layout {\tt scch} with the difference
that it does {\em not} use Verdana. It uses Helvetica instead, so
files using the {\tt scch} layout should run smoothly on every
up-to-date \LaTeX\ installation, regardless of whether Verdana is
available or not.

\subsubsection{Layout \texttt{oldscch}}

Internally, the colors \verb+scchre+ (ruby red), \verb+scchye+
(``\"Osterreichische Post'' yellow), \verb+scchgr+ (medium gray),
\verb+scchwh+ (white), and \verb+scchbl+ (black) are pre-defined,
corresponding to the standard colors of the previous SCCH corporate
identity.

\begin{decl}
|\red| \\
|\yellow|
\end{decl}
These commands can be used to switch to the colors which are
standard for highlighting in the corresponding PowerPoint template.
They can be used like font switches (e.g. \verb+\bf+).

\subsubsection{Layout \texttt{uda}}

The colors \verb+udaaz+ (dark blue), \verb+udagr+ (medium gray),
\verb+udabg+ (very light steel blue), and \verb+udabl+ (black) are
pre-defined.

\begin{decl}
|\blue|\\
|\grey|
\end{decl}
As above, these commands can be used to switch to the colors for
highlighting. It can be used in the same way as \verb+\bf+.


\subsubsection{Layout \texttt{udaold}}

The colors \verb+udati+ (similar to moss-green), \verb+udagr+
(medium gray), \verb+udalg+ (light gray), \verb+udawh+ (white), and
\verb+udabl+ (black) are pre-defined.

\begin{decl}
|\green|
\end{decl}
As above, this command can be used to switch to the colors for
highlighting. It can be used in the same way as \verb+\bf+.

\subsection{Layout-specific convenience macros}

In the \verb+scch-slides+ package, the macros described above had
different names, all starting with \verb+\scch+. In order to be
independent of the chosen layout, this has been changed in \PPTeX.
However, the three layouts still define all these macros in order to
maintain backwards compatibility. A presentation that was written
using the \verb+scch-slides+ class should still produce the same
result if compiled with \verb+\documentclass[scch,a4paper]{pptex}+.


\section{Further Caveats}

\begin{enumerate}
\item Animation is a delicate thing; if some arrangement of
\verb+\step+ statements does not work, try different ones until
the document is processed without error message. In particular,
the first \verb+\item+ of a list environment (including
\verb+itemize+, \verb+enumerate+, and \verb+description+) may not
be surrounded by a \verb+\step+ statement.
\item Slides in portrait format (corresponding to the
\verb+slides*+ environment in the \verb+seminar+ class) are yet
supported.
\item When processing a \PPTeX\ document, a lot of
garbage warnings and other messages are produced.
\item The status of the \TeX Power bundle is still experimental,
and so is the status of the \verb+pptex+ document class.
\item The use of more than one layout within one presentation is
currently not supported.
\item Note that you cannot use optional arguments in optional
arguments; bear that in mind when you use the optional argument of
\verb+\titleslide+.
\end{enumerate}

\section{Related Documents}

This document, more or less, only describes the specific features of
the class \verb+pptex.cls +, which makes excessive use of other
packages and classes, such as \verb+seminar.cls+,
\verb+texpower.sty+, and \verb+hyperref.sty+. For detailed
descriptions, see the enclosed manuals.

\section{License Issues}


All files included in this distribution that were written by Ulrich
Bodenhofer are completely free. They may be downloaded, used,
copied, distributed, modified, without any restrictions and without
the prior acceptance of the author.

However, this distribution also contains material from other
authors, in particular, version 0.0.8f of Stephan Lehmke's TeXPower
package. For these files, license restrictions may apply. Please
check the headers of these files for further information.

\appendix
\section{Making TrueType Fonts Usable for \LaTeX\ and
PDF\LaTeX}\label{app:verdana}

In this appendix, we give instructions on how Verdana can be made
available to \LaTeX/PDF\LaTeX. These instructions are based on the
procedure described by Jens Wei\ss
enburg,\footnote{\url{http://www.weissenburger.de/content/latex5/}}
an update (and German translation) of the method previously
described by Damir
Rakityansky.\footnote{\url{http://www.radamir.com/tex/}} This
procedure has worked perfectly on the author's system and has {\em
not} been tested on other systems. So the author does, in no sense,
guarantee that the procedure described in the following will work on
other systems.

\subsection*{Prerequisites}

Your system needs \TeX\ tools for handling TrueType fonts, such as,
{\tt ttf2tfm}, {\tt ttf2pk}, etc. These programs are included in
recent versions of Mik\TeX\ and also in distributions for other platforms.
Further make sure that the file \verb+T1-WGL4.enc+ is found on your system (search in the
\verb+ttf2tfm+ sub-folder of your (local) \verb+texmf+ directory. If
it is not there, download it from the Internet.

\subsection*{Creating the font-specific files}

Start from an empty folder and copy the relevant TTF files to this folder.
For Verdana, these are the files \verb+verdana.ttf+ (regular), \verb+verdanab.ttf+ (bold),
\verb+verdanai.ttf+ (italic), and \verb+verdanaz.ttf+ (bold italic). For Arial Black,
this is the file \verb+ariblk.ttf+. Open a command/shell window and
change to this directory containing the copy/copies of the
TrueType file(s). 

For producing the Verdana files, enter the following 12 commands (or put them into a shell script/batch program):
\begin{quote}
{\footnotesize\begin{verbatim}
ttf2tfm verdana.ttf -q -T T1-WGL4.enc -v ecvrd.vpl recvrd.tfm >ttfonts.map
ttf2tfm verdanab.ttf -q -T T1-WGL4.enc -v ecvrdb.vpl recvrdb.tfm >>ttfonts.map
ttf2tfm verdanai.ttf -q -T T1-WGL4.enc -v ecvrdi.vpl recvrdi.tfm >>ttfonts.map
ttf2tfm verdanaz.ttf -q -T T1-WGL4.enc -v ecvrdbi.vpl recvrdbi.tfm >>ttfonts.map
ttf2tfm verdana.ttf -q -T T1-WGL4.enc -s .167 -v ecvrdo.vpl recvrdo.tfm >>ttfonts.map
ttf2tfm verdanab.ttf -q -T T1-WGL4.enc -s .167 -v ecvrdbo.vpl recvrdbo.tfm >>ttfonts.map

vptovf ecvrd.vpl ecvrd.vf ecvrd.tfm
vptovf ecvrdb.vpl ecvrdb.vf ecvrdb.tfm
vptovf ecvrdi.vpl ecvrdi.vf ecvrdi.tfm
vptovf ecvrdbi.vpl ecvrdbi.vf ecvrdbi.tfm
vptovf ecvrdo.vpl ecvrdo.vf ecvrdo.tfm
vptovf ecvrdbo.vpl ecvrdbo.vf ecvrdbo.tfm
\end{verbatim}}
\end{quote}

For producing the Arial Black files, enter the following commands (or put them into a shell script/batch program):
\begin{quote}
{\footnotesize\begin{verbatim}
ttf2tfm ariblk.ttf -q -T T1-WGL4.enc -v ecarb.vpl recarb.tfm >ttfonts.map
vptovf ecarb.vpl ecarb.vf ecarb.tfm
\end{verbatim}}
\end{quote}

Note that it is essential that you use exactly the same file names
for the destination files,\NOTE{} otherwise the font definitions in
the files \verb+t1tvrd.fd+ and \verb+t1tarb.fd+ that are supplied with PP\TeX\ will not work.

After having done this, you can delete all files ending with
\verb+.vpl+.

\subsection*{Installing the files such that \TeX\ finds them}

\begin{enumerate}
\item Place all \verb+.tfm+ files in a folder, where \TeX\ searches for \verb+.tfm+ files;
recommended: create a sub-folder of \verb+fonts/tfm/+ in your
\verb+localtexmf+ directory and move the files there.
\item Place the \verb+.vf+ files in a folder, where \TeX\ searches for \verb+.vf+ files;
recommended: create a sub-folder of \verb+fonts/vf/+ in your
\verb+localtexmf+ directory and move the files there.
\item Place the \verb+.ttf+ files in a folder, where \verb+ttf2pk+ can find them. I do not know enough about this,
but at least for me \verb+localtexmf/fonts/ttf+ worked.
\item Append the lines of the above created \verb+ttfonts.map+
file to your system-wide \verb+ttfonts.map+ file (most probably in
the \verb+ttf2tfm/base+ folder in your \verb+texmf+ directory)
\item Once again, make sure that the file \verb+T1-WGL4.enc+ is correctly found on your
system.
\item Refresh the file name database.
\end{enumerate}

\subsection*{Caveats}
\begin{itemize}
\item The fonts' resolution may not be too good. If you zoom into the resulting PDFs,
you will see that they are actually quite edgy. For screen/beamer
presentations, however, the anti-aliasing capabilities of up-to-date
PDF viewers are sufficient not to make the low resolution appear too
annoying.
\item As you may have seen from the output of the \verb+ttf2tfm+
commands above, not all special characters can be translated. In
particular, some ligatures (e.g.\ ff vs.\ f\/f) are not resolved and
are treated as single characters by the new font.
\end{itemize}

\end{document}
