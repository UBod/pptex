%-----------------------------------------------------------------------------------------------------------------
% File: __TPFAQ.tex
%
% Code for the FAQ list of the TeXPower bundle.
% 
% This file is input by others. Don't compile it separately.
%
%-----------------------------------------------------------------------------------------------------------------
% Autor: Stephan Lehmke <Stephan.Lehmke@cs.uni-dortmund.de>
%
% v0.0.1 May 31, 2000: First version for the pre-alpha release of TeXPower.
%
% v0.0.2 Jun  9, 2000: Some additions in the `problems' and `how to' section.
%
% v0.0.3 Jun 15, 2000: Additions in the `usage', `how to', and `problems' sections.
%
% v0.0.4 Jun 22, 2000: Additions in the `usage' and `problems' section.
%


\DefineVerbatimEnvironment{LaTeXCode}{Verbatim}
{gobble=2,formatcom=\codeswitch,frame=single,numbers=left,xleftmargin=1em,numbersep=.5em}

%-----------------------------------------------------------------------------------------------------------------
%
\makeslidetitle
{%
  Frequently asked questions list%
  \thanks{FAQ v0.0.4 of Jun 22, 2000 for \TeX Power v0.0.8f of Jun 27, 2000 (pre-alpha).}%
  }%

\setcounter{firststep}{0}

\tableofcontents

\newslide

%-----------------------------------------------------------------------------------------------------------------
%
\section{General}
\subsection{Where can I get the newest version of the \TeX Power FAQ\,?}
You can download the latest version of the \TeX Power FAQ from the following URLs:
\begin{center}
  \newlength{\urlwidth}%
  \setlength{\urlwidth}{\linewidth-\widthof{Printout version}-2\tabcolsep}%
  \begin{tabular}{@{}p{\urlwidth}l@{}}
    \url{http://Ls1-www.cs.uni-dortmund.de/~lehmke/texpower/doc/FAQ-display.pdf}&Screen version\\
    \url{http://Ls1-www.cs.uni-dortmund.de/~lehmke/texpower/doc/FAQ-printout.pdf}&Printout version
  \end{tabular}%
\end{center}

\newslide

\subsection{What is \TeX Power\,?}
The \TeX Power bundle contains style and class files for creating dynamic online presentations with \LaTeX. 

The heart of the bundle is the package \code{texpower.sty} which implements some commands for presentation effects. This
includes setting page transitions, color highlighting and displaying pages incrementally.

The document class \code{powersem.cls} is a wrapper for seminar which sets up everything for dynamic presentations. 


\newslide

\subsection{Where can I obtain \TeX Power\,?}
The complete bundle, together with its documentation, can be found under the URL
\begin{center}
  \url{http://Ls1-www.cs.uni-dortmund.de/~lehmke/texpower}
\end{center}

For the current \emph{pre-alpha} release, the license forbids any form of redistribution, so for the time being, if you
want a new version of \TeX Power, you have to download it from there.

\newslide

\subsection{Where can I discuss \TeX Power or ask for help\,?}
Bug and problem reports should go to the author \href{mailto:Stephan.Lehmke@cs.uni-dortmund.de}{\name{Stephan Lehmke}}
by email.

Discussions about \TeX Power should take place on the mailing list
\href{mailto:texpower@ls6.cs.uni-dortmund.de}{texpower@ls6.cs.uni-dortmund.de}. To subscribe, send an email message to
the address
\href{mailto:texpower-request@ls6.cs.uni-dortmund.de?subject=subscribe}{texpower-request@ls6.cs.uni-dortmund.de} with
the \concept{subject} \code{subscribe}.

The mailing list is publicly archived at

\url{http://www.mail-archive.com/texpower@ls6.cs.uni-dortmund.de/}

Until the first \emph{alpha} release, I'd like to keep \TeX Power out of the `general' \TeX{} mailing lists and
newsgroups as much as possible, to avoid drawing too much premature attention.


\newslide

\subsection{What alternatives are there to using \TeX Power\,?}
The most prominent alternative to \TeX Power is the \concept{Pdf Presentation Post Processor} PPower4, the homepage of
which is
\begin{quote}
  \url{http://www-sp.iti.informatik.tu-darmstadt.de/software/ppower4/}
\end{quote}

Another alternative is the \concept{Utopia PDF Presentations Bundle}, which provides a complete presentation design
environment. Its home page is
\begin{quote}
  \url{http://www.utopiatype.com.au/products/ubundle.html}
\end{quote}

Comparisons of different presentation packages can be found on the home page of \name{Prof.\,D.\,P.\,Story}:
\begin{quote}
  \url{http://www.math.uakron.edu/~dpstory/pdf_demos.html}
\end{quote}
and in the talk held by \name{Ross Moore} at the \name{California Institute of Technology} on 8th May 2000:
\begin{quote}
  \url{http://www.cds.caltech.edu/caltex/2000/}
\end{quote}
 




\newslide

%-----------------------------------------------------------------------------------------------------------------
%
\section{Usage}

\subsection{How do I design a presentation with \TeX Power\,?}
It should be stressed that \TeX Power is \underl{not} (currently) a complete presentation package. It just adds dynamic
presentation effects (and some other gimmicks specifically interesting for dynamic presentations) and should always be
combined with a document class dedicated to designing presentations (or a package like
\href{ftp://ftp.dante.de/tex-archive/help/Catalogue/entries/pdfslide.html}{\code{pdfslide}}).

There are demos in the \href{http://Ls1-www.cs.uni-dortmund.de/~lehmke/texpower/doc}{\code{doc}} directory for most
popular presentation-making document classes and packages.


\newslide

\subsection{I find \TeX Power very complicated. How can I learn how to realize dynamic effects\,?}

As always with \TeX, you should first make up your mind what kind of effect you desire, and what \LaTeX{} structures
will be involved.

Then you should check the examples in the \href{http://Ls1-www.cs.uni-dortmund.de/~lehmke/texpower/doc}{\code{doc}}
directory for anything similar to what you want. If you find anything suitable, read the corresponding code. There are
some inline comments to explain what's going on. Print out the
\href{http://Ls1-www.cs.uni-dortmund.de/~lehmke/texpower/doc/manual.pdf}{\code{manual}} for documentation of the \TeX
Power commands.

Further `recepies' can be found in section \ref{Sec:HowTo}.

If you don't find anything suitable you can modify to your needs, and can't figure out from the documentation how to
achieve your aims, please report to \href{mailto:Stephan.Lehmke@cs.uni-dortmund.de}{\name{Stephan Lehmke}} by email. If
you've found an application for \TeX Power not covered by the examples, a new example should be created.


\newslide

\subsection{Can I combine \TeX Power with PPower4\,?}
There is no problem postprocessing documents in which \TeX Power is used. This can be useful, for instance, for
realising structured backgrounds with the \code{background} package from the
\href{http://www-sp.iti.informatik.tu-darmstadt.de/software/ppower4/}{PPower4 bundle}.

If there are presentation effects for which you'd like to use PPower4's implementation of the \macroname{pause} command,
then just load PPower4's \code{pause} package. PPower4's definition of \macroname{pause} will override
\code{texpower}'s. Then you can combine PPower4's \macroname{pause} functionality with \TeX Power's \macroname{stepwise}
functionality, for maximum expressive power.

\newslide

\subsection{I'm missing some of the classes and packages used in the demo and example files.}

First of all, it has to be said that \TeX Power makes use of some `modern' features which have been introduced into the
\TeX{} System quite recently and are evolving swiftly. The core of the \code{texpower} package, namely the commands
\macroname{pause} and \macroname{stepwise} is implemented in `pure' \LaTeX{} and should be largely independent of any
fancy extensions, but to get most out of \TeX Power's presentation features and process the more advanced examples, it
is recommended to have a moderately new \TeX{} distribution installed (rule of thumb: not older than one year).

But even if your distribution is quite new, it might not contain some of the classes and packages used by the demos and
examples. Here's a list of (hopefully all of) the packages and classes used (which are not part of core \LaTeX) and
their availability:


\newlength{\twidth}%
{%
  \makeatletter
  \ifthenelse{\boolean{display}}{\def\@outputpage{\setbox\@cclv\box\@outputbox\slide@output}}{}%
  \makeatother
  \setlength{\twidth}{\linewidth-\widthof{\code{pdfscreen}}-2cm-6\tabcolsep}%
\begin{longtable}{lp{2cm}>{\raggedright\footnotesize}p{\twidth}}
  Package&used in&\normalsize available from\tabularnewline\hline\endhead
  \href{ftp://ftp.dante.de/tex-archive/help/Catalogue/entries/hyperref.html}{\code{hyperref}}&most
  &\href{http://www.ctan.org/}{CTAN}, e.\,g.\
  \url{ftp://ftp.dante.de/tex-archive/macros/latex/contrib/supported/hyperref}
  \tabularnewline
  \href{ftp://ftp.dante.de/tex-archive/help/Catalogue/entries/url.html}{\code{url}}&most
  &\href{http://www.ctan.org/}{CTAN}, e.\,g.\ \url{ftp://ftp.dante.de/tex-archive/macros/latex/contrib/other/misc}
  \tabularnewline
  \href{ftp://ftp.dante.de/tex-archive/help/Catalogue/entries/soul.html}{\code{soul}}&many
  &\href{http://www.ctan.org/}{CTAN}, e.\,g.\ \url{ftp://ftp.dante.de/tex-archive/macros/latex/contrib/supported/soul}
  \tabularnewline
  \href{ftp://ftp.dante.de/tex-archive/help/Catalogue/entries/pstricks.html}{\code{pstricks}}
  &\small\code{fulldemo}, \code{picexample}
  &\href{http://www.ctan.org/}{CTAN}, e.\,g.\ \url{ftp://ftp.dante.de/tex-archive/graphics/pstricks}
  \tabularnewline
  \href{ftp://ftp.dante.de/tex-archive/help/Catalogue/entries/xr.html}{\code{xr-hyper}}
  &\small\code{manual}
  &\href{http://www.ctan.org/}{CTAN}, e.\,g.\
  \url{ftp://ftp.dante.de/tex-archive/macros/latex/contrib/supported/hyperref} 
  \tabularnewline
  \href{ftp://ftp.dante.de/tex-archive/help/Catalogue/entries/fancyvrb.html}{\code{fancyvrb}}
  &\small\code{FAQ}
  &\href{http://www.ctan.org/}{CTAN}, e.\,g.\
  \url{ftp://ftp.dante.de/tex-archive/macros/latex/contrib/supported/fancyvrb} 
  \tabularnewline
  \href{ftp://ftp.dante.de/tex-archive/help/Catalogue/entries/pdfscreen.html}{\code{pdfscreen}}
  &\small\code{pdfscrdemo}
  &\href{http://www.ctan.org/}{CTAN}, e.\,g.\
  \url{ftp://ftp.dante.de/tex-archive/macros/latex/contrib/supported/pdfscreen} 
  \tabularnewline
  \href{ftp://ftp.dante.de/tex-archive/help/Catalogue/entries/pdfslide.html}{\code{pdfslide}}
  &\small\code{pdfslidemo}
  &\href{http://www.ctan.org/}{CTAN}, e.\,g.\
  \url{ftp://ftp.dante.de/tex-archive/macros/latex/contrib/supported/pdfslide} 
  \tabularnewline
  \href{ftp://ftp.dante.de/tex-archive/help/Catalogue/entries/ppower4.html}{\code{pp4slide}}
  &\small\code{pp4sldemo}
  &\href{http://www.ctan.org/}{CTAN}, e.\,g.\
  \url{ftp://ftp.dante.de/tex-archive/support/ppower4/pp4sty.zip} 
  \tabularnewline
  \href{ftp://ftp.dante.de/tex-archive/help/Catalogue/entries/ifmslide.html}{\code{ifmslide}}
  &\small\code{ifmslidemo}
  &\href{http://www.ctan.org/}{CTAN}, e.\,g.\
  \url{ftp://ftp.dante.de/tex-archive/macros/latex/contrib/supported/ifmslide} 
  \tabularnewline
\end{longtable}
}
  
{%
  \makeatletter
  \ifthenelse{\boolean{display}}{\def\@outputpage{\setbox\@cclv\box\@outputbox\slide@output}}{}%
  \makeatother
  \setlength{\twidth}{\linewidth-\widthof{\code{scrartcl}}-2cm-6\tabcolsep}%
\begin{longtable}{lp{2cm}>{\raggedright\footnotesize}p{\twidth}}
  Class&used in&\normalsize available from\tabularnewline\hline\endhead
  \href{ftp://ftp.dante.de/tex-archive/help/Catalogue/entries/seminar.html}{\code{seminar}}&most
  &\href{http://www.ctan.org/}{CTAN}, e.\,g.\
  \url{ftp://ftp.dante.de/tex-archive/macros/latex/contrib/other/seminar}
  \tabularnewline
  \href{ftp://ftp.dante.de/tex-archive/help/Catalogue/entries/koma-script.html}{\code{scrartcl}}&most
  &\href{http://www.ctan.org/}{CTAN}, e.\,g.\
  \url{ftp://ftp.dante.de/tex-archive/macros/latex/contrib/supported/koma-script}
  \tabularnewline
  \href{ftp://ftp.dante.de/tex-archive/help/Catalogue/entries/foiltex.html}{\code{foils}}
  &\small\code{foilsdemo}, \code{pp4sldemo}
  &\href{http://www.ctan.org/}{CTAN}, e.\,g.\
  \url{ftp://ftp.dante.de/tex-archive/nonfree/macros/latex/contrib/supported/foiltex}
  \tabularnewline
\end{longtable}
}


\newslide

The \TeX Power bundle is in some sense `intertwined' with some packages and files which develop very fast at the moment,
because it builds on features provided by those. Consequently, the development of \TeX Power has to be `synchronised'
with the development of these packages. To obtain the best results with \TeX Power, it might be advisable to download
the newest version of the following files from their home page (which is likely to be newer than the newest version on
CTAN):

{%
  \makeatletter
  \ifthenelse{\boolean{display}}{\def\@outputpage{\setbox\@cclv\box\@outputbox\slide@output}}{}%
  \makeatother
  \setlength{\twidth}{\linewidth-\widthof{File/Package}-4\tabcolsep}%
  \begin{longtable}{l>{\raggedright\footnotesize}p{\twidth}}
    File/Package&\normalsize available from\tabularnewline\hline\endhead
    \code{hyperref}&\url{http://www.tug.org/applications/hyperref/hyperref.zip}\tabularnewline
    \code{pdftex.def}&\url{http://www.tug.org/applications/pdftex/pdftex.def}\tabularnewline
  \end{longtable}
  }


\newslide

%-----------------------------------------------------------------------------------------------------------------
%
\section{How do I\dots}\label{Sec:HowTo}

\subsection{How can I incrementally display a paragraph of text\,?}\label{Q:Par}

The easiest solution is to use \macroname{parstepwise}, but if the arguments of \macroname{step} are long, you'll get
problems with line breaks, as \macroname{parstepwise} forces \macroname{step} to put its argument in a box.

You can use \macroname{hidetext} like this:

\begin{LaTeXCode}
  \stepwise[\let\hidestepcontents=\hidetext]
  {\step{Line breaks} \step{work in here.}}
\end{LaTeXCode}


\stepwise[\let\hidestepcontents=\hidetext]
{%
  \ifthenelse{\boolean{display}}
  {%
    yields
    \present
    {%
      \begin{minipage}{7em}
        \step{Line breaks} \step{work in here.}
      \end{minipage}%
      }
    }
  {}
  
  But note that \macroname{hidetext}, being implemented using the \code{soul} package, is quite fragile (compare
  \ref{Sec:hidetext}).

  }

\newslide

If you're not using structured backgrounds, \macroname{hidevanish} is another alternative which can be used exactly like
\macroname{hidetext}, but is much more robust (note that this will fail whenever your text should appear in front of
different background colors, for any reason).

In the argument of \macroname{hidevanish}, which uses \macroname{textcolor}, paragraph breaks are not allowed. Using
\macroname{vstep} is a little less restrictive:


\begin{LaTeXCode}
  \stepwise
  {%
    {\vstep Line and paragraph breaks 
    \vstep work in here.\par Yeah!}%
    }
\end{LaTeXCode}

\stepwise[\renewcommand{\vanishcolor}{presentcolor}]
{%
  \ifthenelse{\boolean{display}}
  {%
    yields
    \present
    {%
      \begin{minipage}{13em}
        \vstep Line and paragraph breaks \vstep work in here.\par Yeah!
      \end{minipage}%
      }
    }
  {}
}

\newslide

To facilitate the decision, here's a side-by-side comparison of the pros and cons:

\macroname{parstepwise}:
\begin{itemize}
\item[\origmath{+}] robust
\item[\origmath{+}] works with structured backgrounds
\item[\origmath{-}] no automatic line breaks in \macroname{step}'s argument 
\item[\origmath{-}] no paragraph breaks in \macroname{step}'s argument 
\end{itemize}

\macroname{hidetext}:
\begin{itemize}
\item[\origmath{-}] very fragile
\item[\origmath{+}] works with structured backgrounds
\item[\origmath{+}] allows automatic line breaks in \macroname{step}'s argument 
\item[\origmath{-}] no paragraph breaks in \macroname{step}'s argument 
\end{itemize}

\newslide

\macroname{hidevanish}:
\begin{itemize}
\item[\origmath{+}] robust
\item[\origmath{-}] fails with structured backgrounds
\item[\origmath{+}] allows automatic line breaks in \macroname{step}'s argument 
\item[\origmath{-}] no paragraph breaks in \macroname{step}'s argument 
\end{itemize}

\macroname{vstep}:
\begin{itemize}
\item[\origmath{+}] very robust
\item[\origmath{-}] fails with structured backgrounds
\item[\origmath{+}] allows automatic line breaks
\item[\origmath{+}] allows paragraph breaks
\end{itemize}

\newslide


\subsection{Instead of making text appear `out of nowhere', I'd rather just change colors from `dimmed' to normal.}

There are some analogies between this item and \ref{Q:Par}.

If you're using \code{texpower}'s standard colors, probably \macroname{hidedimmed} does what you want:

\begin{LaTeXCode}
  \stepwise[\let\hidestepcontents=\hidedimmed]
  {%
    \step{This works with} \step{\emph{all}} 
    \step{\highlighttext{highlighting} commands.}%
    }
\end{LaTeXCode}

\stepwise[\let\hidestepcontents=\hidedimmed]
{%
  \ifthenelse{\boolean{display}}
  {%
    yields
    \present
    {%
      \begin{minipage}{10em}
        \step{This works with} \step{\emph{all}} \step{\highlighttext{highlighting} commands.}
      \end{minipage}%
      }
    }
  {}
  }

\newslide

In the argument of \macroname{hidedimmed}, which uses \macroname{textcolor}, paragraph breaks are not allowed. Using
\macroname{dstep} is a little less restrictive. The following achieves the same result as above:

\begin{LaTeXCode}
  \stepwise
  {%
    \dstep This works with \dstep \emph{all}
    \dstep \highlighttext{highlighting} commands.%
    }
\end{LaTeXCode}

\newslide

If the dimmed colors look too fancy to you, you can also use \macroname{vstep} for this purpose, setting
\macroname{vanishcolor} to some `dimmed' color:

\begin{LaTeXCode}
  \stepwise[\renewcommand{\vanishcolor}{inactivecolor}]
  {%
    \vstep This works with \vstep \emph{all}
    \vstep \highlighttext{highlighting} commands.%
    }
\end{LaTeXCode}

\stepwise[\renewcommand{\vanishcolor}{inactivecolor}]
{%
  \ifthenelse{\boolean{display}}
  {%
    yields
    \present
    {%
      \begin{minipage}{10em}
        \vstep This works with \vstep \emph{all} \vstep \highlighttext{highlighting} commands.
      \end{minipage}%
      }
    }
  {}
  }

Achieving the same with \macroname{hidevanish} is left as an exercise to the reader.

\newslide

\subsection{\macroname{dstep} and \macroname{hidedimmed} work only with \code{texpower}'s standard colors. How can I dim
  my own colors\,?} 

\code{texpower} maintains a list of colors which will be affected by \macroname{dimcolors} (which is behind
\macroname{dstep} and \macroname{hidedimmed}). 

You can add your own colors to this list by issuing \commandapp{addTPcolor}{mycolor}. Then you only have to define
another color \code{dmycolor} which will be replaced for \code{mycolor} automatically when \macroname{dimcolors} is
executed.

\newslide

For instance:
{\small
\begin{LaTeXCode}
  \definecolor{mycolor}{rgb}{1,0.5,0}%
  \definecolor{dmycolor}{rgb}{0.9,0.8,0.6}%
  \addTPcolor{mycolor}
  \stepwise
  {\dstep My \emph{own} \dstep \textcolor{mycolor}{color}.}
\end{LaTeXCode}
}%

\ifthenelse{\boolean{display}}
{%
  \definecolor{mycolor}{rgb}{1,0.5,0}%
  \definecolor{dmycolor}{rgb}{0.9,0.8,0.6}%
  \addTPcolor{mycolor}%
  \liststepwise
  {%
    yields
    \present
    {%
      \dstep My \emph{own} \dstep \textcolor{mycolor}{color}.
      }
    }
  }
{}

Note that if you ever wish to use \macroname{enhancecolors} or \macroname{highlightenhanced}, you'll also need an
\emph{enhanced} version of your new color named \code{emycolor}.

If you wish to use one of the commands \macroname{whitebackground}, \macroname{lightbackground},
\macroname{darkbackground}, or \macroname{blackbackground}, you'll need even more variants of your new color. In this
case, you'll better define it in the file \code{tpsettings.cfg} (which contains an example).


\newslide

%-----------------------------------------------------------------------------------------------------------------
%
\section{Problems}

\subsection{I'm loading the \code{texpower} package, but dynamic features don't seem to work.}

Remember that you have to turn on dynamic features explicitly by giving the \code{display} option either to
\code{texpower} or as a global option. Otherwise, a printout version of your document is produced.

\newslide

\subsection{When I use the \code{colormath} option, my displayed formulae are not colored.}

Don't use the \TeX{} environment \code{\$\$}\dots\code{\$\$} for displayed formulae if you want to profit from math
coloring.

\code{texpower} supports \LaTeX's environments \macroname{[}\dots\macroname{]}, \code{displaymath}, \code{equation},
\code{eqnarray}, and \code{eqnarray*}. It also works with the diverse displayed math environments from the
\href{ftp://ftp.dante.de/tex-archive/help/Catalogue/entries/amsmath.html}{\code{amsmath}} package.

Replacing \code{\$\$}\dots\code{\$\$} everywhere by \macroname{[}\dots\macroname{]} should solve this problem.

\newslide

\subsection{It seems I can't use \macroname{vfill} in combination with \macroname{pause}.}

This is a problem indeed, as \LaTeX{} never gets to see anything after \macroname{pause} when the first part of the
sequence is produced. You can use \macroname{vfill} with \macroname{stepwise} if you
\begin{enumerate}
\item use a configuration where \macroname{step} leaves blank space (to ensure proper vertical spacing);
\item put \emph{all} \macroname{vfill}s into the argument of \macroname{stepwise}, \emph{outside} the argument of any
  \macroname{step}.
\end{enumerate}

For instance:

\begin{LaTeXCode}
  \parstepwise
  {\step{One.}\vfill\step{Two.}\vfill\step{Three.}}
\end{LaTeXCode}

\parstepwise
{%
  \ifthenelse{\boolean{display}}
  {%
    yields
    \present
    {%
      \begin{minipage}[c][10ex][s]{10em}
        \step{One.}\vfill\step{Two.}\vfill\step{Three.}
      \end{minipage}
      }
    }
  {}
  }


\newslide

\subsection{When I use \LaTeX+\code{dvips}+\code{distiller}, the result looks strange and `hyper' features don't work.}

Check the \code{log} file of your document. If it contains the line
{\codeswitch%
\begin{verbatim}
*hyperref using default driver hypertex*
\end{verbatim}%
}%
then the default hyperref driver for your system is not suited for processing by \code{dvips}+\code{distiller}.

Either you set another default driver (for instance, in the file \code{hyperref.cfg}), or you use the option
\code{dvips} in your document as a global option or an option to \commandapp{usepackage}{hyperref}. See the
documentation of the \href{ftp://ftp.dante.de/tex-archive/help/Catalogue/entries/hyperref.html}{\code{hyperref}} package
for details.

\newslide

\subsection{When using \macroname{highlighttext} or \macroname{hidetext}, I'm getting strange error messages.}
\label{Sec:hidetext}

Note that both these commands are implemented using the
\href{ftp://ftp.dante.de/tex-archive/help/Catalogue/entries/soul.html}{\code{soul}} package. \code{soul} has some rather
severe restrictions concerning what is allowed to appear in the argument of commands using it. Consult the documentation
of \href{ftp://ftp.dante.de/tex-archive/help/Catalogue/entries/soul.html}{\code{soul}} for a detailed description of
these restrictions.

The most prominent one is that almost no \LaTeX{} command is allowed in the argument of a command implemented using
\code{soul}. For instance, to use an emphasis or highlighting command like \macroname{emph}, you have to use a sequence
of \macroname{highlighttext} commands, putting \macroname{emph} `outside'. Expect glitches in display quality though.

\code{\small\commandapp{highlighttext}{This~}\discretionary{\%}{}{}\commandapp{emph}{\commandapp{highlighttext}{annoying~}}\discretionary{\%}{}{}\commandapp{highlighttext}{behaviour}}
yields \highlighttext{This }\emph{\highlighttext{annoying }}\highlighttext{behaviour}.

\newslide

Another restriction is that accents are separated from the characters they belong to and break. You have to enclose the
complete accented character with braces or use an appropriate input encoding, typing accented characters `as one'.
\begin{center}
  \begin{tabular}{l@{ yields }l}
    \commandapp{highlighttext}{S\{\macroname{"}u\}\macroname{ss} es}&\highlighttext{S{\"u}\ss es}\\
    \commandapp{highlighttext}{S��es}&\highlighttext{S��es}\\
  \end{tabular}
\end{center}

\newslide

\subsection{Inside the argument of \macroname{stepwise}, all counters seem to be freezed on all pages of the sequence
  generated. How can I use a self-defined counter which does not freeze\,?}

Freezing counters is a desirable behaviour in general, for instance to stop equation numbers from going astray.

But \code{texpower} maintains a list of counters which are \emph{not} freezed, containing for instance the counter
\code{step}. 

If you need a counter for special effects while the incremental sequence is generated (for instance: generating a
sequence of MetaPost figures with the \code{emp} and \code{feynmp} packages), use 
\begin{LaTeXCode}
  \releasecounter{mycounter}
\end{LaTeXCode}
to release the counter \code{mycounter}.

 
%%% Local Variables: 
%%% mode: latex
%%% fill-column: 120
%%% TeX-master: "FAQ-display"
%%% End: 
