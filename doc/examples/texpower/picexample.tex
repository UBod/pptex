%-----------------------------------------------------------------------------------------------------------------
% File: picexample.tex
%
% Picture example for the package texpower.sty.
% 
% This file can be compiled with pdfLaTeX or (standard) LaTeX. When using standard LaTeX, the dvi file produced should
% be processed with 
% 
% dvips -Ppdf -j0 picexample
% 
% afterwards processing the resulting ps file with
%
% distill picexample.ps
%
% (The syntax is for a unix system with tetex 1.0 and distiller 3. Modify appropriately for other configurations.)
%
% The resulting pdf file is meant for presenting `interactively' with Adobe Acrobat Reader.
%
% Note that different pictures are produced with pdflatex and latex, because pdflatex does not support PSTricks.
%
%-----------------------------------------------------------------------------------------------------------------
% Autor: Stephan Lehmke <Stephan.Lehmke@cs.uni-dortmund.de>
%
% v0.0.1 Mar 20, 2000: First version for the pre-alpha release of TeXPower.
%
% v0.0.2 Apr 27, 2000: Some small changes in preparation of the update to TeXpower v0.0.7.
%

%-----------------------------------------------------------------------------------------------------------------
% Use slifonts and a dark background. 

\PassOptionsToPackage{colormath,colorhighlight,slifonts,darkbackground}{texpower}

% Input the generic preamble.

%-----------------------------------------------------------------------------------------------------------------
% File: __TPpble.tex
%
% Code for the preamble of TeXPower demos.
% 
% This file is input by others. Don't compile it separately.
%
%-----------------------------------------------------------------------------------------------------------------
% Autor: Stephan Lehmke <Stephan.Lehmke@cs.uni-dortmund.de>
%
% v0.0.1 Mar 20, 2000: First version for the pre-alpha release of TeXPower.
%

%-----------------------------------------------------------------------------------------------------------------
% File: __TPpbla.tex
%
% Code for the first part of the preamble of TeXPower demos.
% 
% This file is input by others. Don't compile it separately.
%
%-----------------------------------------------------------------------------------------------------------------
% Autor: Stephan Lehmke <Stephan.Lehmke@cs.uni-dortmund.de>
%
% v0.0.1 Mar 20, 2000: First version for the pre-alpha release of TeXPower.
% v0.0.2 Mar 21, 2000: Remedying an incompatibility between LaTeX releases concerning the implementation of
%                      \@iiiparbox (Apr 11: this code is now part of texpower.sty). 
% v0.0.3 Apr 11, 2000: Color emphasis code moved into texpower.
%

\documentclass
[%
%-----------------------------------------------------------------------------------------------------------------
% Document class options:
% -----------------------
%
% Landscape slides formatted for letter paper fit most screen resolutions (more or less).
%
  letterpaper,%
  landscape,%
% 
% The KOMA option makes powersem load scrartcl.cls instead of article.cls.
%
  KOMA,%
% KOMA document class options are accepted.
  smallheadings,%
%
% The calcdimensions option makes powersem calculate the slide dimensions automatically from paper size and margins.
  calcdimensions,%
%
% The display option sets everything up for producing slides to be displayed interactively.
% This option is also recognized by the texpower package.
%
  display%
%-----------------------------------------------------------------------------------------------------------------
]
%-----------------------------------------------------------------------------------------------------------------
% Document class powersem, based on seminar.cls for simulating ppower via latex+distiller (instead of pdflatex).
%
{powersem}
%-----------------------------------------------------------------------------------------------------------------


%-----------------------------------------------------------------------------------------------------------------
% Set slide margins rather small for maximum use of space. This is a demo, remember.
%
\renewcommand{\slidetopmargin}{5mm}
\renewcommand{\slidebottommargin}{5mm}

\renewcommand{\slideleftmargin}{5mm}
\renewcommand{\sliderightmargin}{5mm}


%-----------------------------------------------------------------------------------------------------------------
% We need some more packages...
%

\usepackage{url}

\usepackage[latin1]{inputenc}

% One more Text emphasis command...

\let\name=\textsc


%%% Local Variables: 
%%% mode: latex
%%% fill-column: 120
%%% TeX-master: nil
%%% End: 

%-----------------------------------------------------------------------------------------------------------------
% File: __TPpblb.tex
%
% Code for the second part of the preamble of TeXPower demos.
% 
% This file is input by others. Don't compile it separately.
%
%-----------------------------------------------------------------------------------------------------------------
% Autor: Stephan Lehmke <Stephan.Lehmke@cs.uni-dortmund.de>
%
% v0.0.1 Mar 20, 2000: First version for the pre-alpha release of TeXPower.
% v0.0.2 Mar 22, 2000: Now loading the config file.
% v0.0.3 Mar 29, 2000: texpower doesn't load hyperref any more; there's now a package fixseminar. 
% v0.0.4 Apr 19, 2000: Added \slidetitle command. 
%


%-----------------------------------------------------------------------------------------------------------------
% We load hyperref and fixseminar which fixes some problems with seminar.
%
\usepackage[bookmarksopen,colorlinks,urlcolor=red,pdfpagemode=FullScreen]{hyperref}
\usepackage{fixseminar}

%-----------------------------------------------------------------------------------------------------------------
% Finally, the texpower package is loaded. 
%
\usepackage{texpower}

%% The configuration file allows user-specific settings.

%-----------------------------------------------------------------------------------------------------------------
% File: __TP.cfg
%
% Code for user-specific configuration of TeXPower documentation files.
% 
% This file is input by others. Don't compile it separately.
%
%-----------------------------------------------------------------------------------------------------------------
% Autor: Stephan Lehmke <Stephan.Lehmke@cs.uni-dortmund.de>
%
% v0.0.1 Mar 22, 2000: First version for the pre-alpha release of TeXPower.
%

\hypersetup{baseurl={http://ls1-www.cs.uni-dortmund.de/\string~lehmke/texpower/doc/}}

\hypersetup{pdfsubject={Documentation and Examples for the texpower package}}

\hypersetup{pdfauthor={Stephan Lehmke}}


%%% Local Variables: 
%%% mode: latex
%%% TeX-master: nil
%%% End: 


%-----------------------------------------------------------------------------------------------------------------
% Some more parameters...
%
\slidesmag{5}
\slideframe{none}
\pagestyle{empty}
\setcounter{tocdepth}{2}

%-----------------------------------------------------------------------------------------------------------------
% The following command produces a title page for every example and documentation file.

\newcommand{\makeslidetitle}[1]
{%
  \title{The \TeX Power bundle\\[2ex]{\normalfont #1}}
  \author{Stephan Lehmke\\\url{mailto:Stephan.Lehmke@cs.uni-dortmund.de}}
  \maketitle

  \newslide
  \setcounter{firststep}{1}% This way, the first step of all examples is displayed.
}

%%% Local Variables: 
%%% mode: latex
%%% fill-column: 120
%%% TeX-master: "dummy"
%%% End: 


%%% Local Variables: 
%%% mode: latex
%%% TeX-master: nil
%%% End: 

\hypersetup{pdftitle={texpower picture example}}


\ifthenelse{\boolean{psspecialsallowed}}% Can we use PSTricks?
{% Yes.
  % PsTricks (sic) is used for creating the picture example.

  \usepackage{pstcol}
  \usepackage{pst-node}
  
  \psset{unit=\unitlength}
  }
{% No. We'll make do without.
  }


%-----------------------------------------------------------------------------------------------------------------
% Finally, everything is set up. Here we go...
%
\begin{document}
\begin{slide}
  %-----------------------------------------------------------------------------------------------------------------
% File: __TPpic.tex
%
% Code for the picture example for the package texpower.sty.
% 
% This file is input by others. Don't compile it separately.
%
%-----------------------------------------------------------------------------------------------------------------
% Autor: Stephan Lehmke <Stephan.Lehmke@cs.uni-dortmund.de>
%
% v0.0.1 Mar 21, 2000: First version for the pre-alpha release of TeXPower.
%
% v0.0.2 Apr 27, 2000: Some small changes in preparation of the update to TeXpower v0.0.7.
%

%-----------------------------------------------------------------------------------------------------------------
%
\makeslidetitle{\macroname{stepwise} Example: A Picture}\label{Sec:ExPic}

\ifthenelse{\boolean{psspecialsallowed}}% Can we use PSTricks?
{%-----------------------------------------------------------------------------------------------------------------
% File: __TPppic.tex
%
% Code for the PSTricks picture example for the package texpower.sty.
% 
% This file is input by others. Don't compile it separately.
%
%-----------------------------------------------------------------------------------------------------------------
% Autor: Stephan Lehmke <Stephan.Lehmke@cs.uni-dortmund.de>
%
% v0.0.1 Mar 21, 2000: First version for the pre-alpha release of TeXPower.
%
% v0.0.2 Apr 19, 2000: Using \bstep instead of \boxedsteps.
%
% v0.0.3 Apr 28, 2000: Some small changes in preparation of the update to TeXpower v0.0.7.
%

%-----------------------------------------------------------------------------------------------------------------
%
% This has nothing to do with \stepwise, just setting up the picture...
%
\newcommand{\Block}[1]
{%
  \begin{pspicture}(-4,-2)(4,2)
    \pspolygon(-4,0)(-2,2)(2,2)(4,0)(2,-2)(-2,-2)
    \rput(0,0){#1}
  \end{pspicture}%
  }%
\ifthenelse{\boolean{TPcolor}}
{\psset{unit=3mm,fillstyle=solid,fillcolor=highlightcolor,linecolor=textcolor}}%
{\psset{unit=3mm}}%
%
% In the following picture, picture items are built incrementally.
%
% \stepwise generates a sequence of slides, all alike. The only difference ist that on every slide, one more of the
% \step commands occurring in the argument of \stepwise are `activated'. This way the stuff inside the argument of
% \stepwise is gone through `step by step'.
%
\stepwise
{%
  \begin{center}
    \large
    \begin{pspicture}(-1,-13)(33,9)
      \rput(0,0){\rnode{x}{\fboxrule0pt\fbox{$x[t]$}}}
      \rput(32,0){\rnode{y}{\fboxrule0pt\fbox{$y(t)$}}}
      {%
        % The effect of the \step command is to `hide' its argument on the first slide of the sequence generated by
        % \stepwise and display it on all other slides of this sequence. The second \step command starts displaying on
        % the third slide and so on...   
        %
        % Usually `hiding' means ignoring altogehter. For the following application however, the box displayed by
        % \step is used as an node in the picture. This means that instead of ignoring its argument, it's better if
        % \step displays an appropriate amount of blank space.
        % This behaviour (create blank space) is exhibited by the command \bstep.
        \rput(4,0){\rnode{V}{\bstep{\psframebox{\Large$V$}}}}
        \ncline{->}{x}{V}
        %
        % The command \rebstep allows both boxes to appear simultaneously.
        %
        \rput(28,0){\rnode{plus}{\rebstep{\psframebox{\Large$+$}}}}
        }%
      \ncline{->}{plus}{y}
      % By using \restep again, the two boxes defining the operators appear at the same time as the first block of the
      % diagram, which is produced by the following commands.
      \restep
      {%
        % We use \afterstep{\pageTransitionDissolve} to make this first step of the diagram appear with a fancy page
        % transition. We have to use \afterstep because the page transition setting would be undone by the group
        % closing contained in \end{pspicture}.
        \afterstep{\pageTransitionDissolve}%
        \rput(16,0){\rnode{IBlk}{\Block{I-Block}}}
        \ncline{->}{V}{IBlk}
        \Aput{$x[t]$}
        \ncline{->}{IBlk}{plus}
        \Aput
        {%
          %
          % Here, one use of \step is nested inside the other. 
          $%
          % Note how the math spacing is corrected manually by adding {} after \cdot. Otherwise, \cdot wouldn't be
          % aware that something is following and act as a postfix (instead of infix) operator.
          % \afterstep is used again to reset the page transition to Replace for the building of the formula.
          \bstep{\afterstep{\pageTransitionReplace}{}b\cdot{}}%
          \bstep{\int\limits^t_0 x(\tau)\,d\tau}%
          $%
          }%
        }%
      \step
      {%
        \afterstep{\pageTransitionDissolve}%
        \rput(16,6){\rnode{PBlk}{\Block{P-Block}}}
        \ncangle[angleA=90,angleB=180,fillstyle=none]{->}{V}{PBlk}
        \Aput{$x(t)$}
        \ncangle[angleB=90,fillstyle=none]{->}{PBlk}{plus}
        \aput(.5){$a\cdot x(t)$}
        }%
      \step
      {%
        \rput(16,-6){\rnode{DBlk}{\Block{D-Block}}}
        \ncangle[angleA=-90,angleB=180,fillstyle=none]{->}{V}{DBlk}
        \Aput{$x(t)$}
        \ncangle[angleB=-90,fillstyle=none]{->}{DBlk}{plus}
        \aput(.5){$\displaystyle c\cdot \left(\frac{dx}{d\tau}\right)(t)$}
        }
    \end{pspicture}%
  \end{center}
  }%

% The whole execution of \stepwise is encapsuled in a group to make all changes local. This affects also the setting
% of the page transition to dissolve in the last but one step. We set it again explicitly to make the last step appear
% with this page transition as well.

\pageTransitionDissolve


%%% Local Variables: 
%%% mode: latex
%%% fill-column: 120
%%% TeX-master: "picexample"
%%% End: 
}{%-----------------------------------------------------------------------------------------------------------------
% File: __TPlpic.tex
%
% Code for the LaTeX picture example for the package texpower.sty.
% 
% This file is input by others. Don't compile it separately.
%
%-----------------------------------------------------------------------------------------------------------------
% Autor: Stephan Lehmke <Stephan.Lehmke@cs.uni-dortmund.de>
%
% v0.0.1 Mar 21, 2000: First version for the pre-alpha release of TeXPower.
%
% v0.0.2 Apr 19, 2000: Using \bstep instead of \boxedsteps.
%
% v0.0.3 Apr 28, 2000: Some small changes in preparation of the update to TeXpower v0.0.7.
%

  %
  % This has nothing to do with \stepwise, just setting up the picture...
  %
  \newcommand{\leftm}{\left\{\!\!\left\{}
  \newcommand{\rightm}{\right\}\!\!\right\}}
  \newcommand{\cls}[3]{\left[#1,#2,#3\right]}
  \newcommand{\WT}[2]{\fbox{${#1}#2$}}
  \newcommand{\GEQ}[1]{\WT{\geq}{#1}}

  %
  % In the following picture, picture items are built incrementally.
  %
  % \stepwise generates a sequence of slides, all alike. The only difference ist that on every slide, one more of the
  % \step commands occurring in the argument of \stepwise are `activated'. This way the stuff inside the argument of
  % \stepwise is gone through `step by step'.
  %
  {%
    \setlength{\unitlength}{1.35\semcm}%
    \footnotesize%
    \setlength{\fboxsep}{1.5pt}%
    \stepwise
    {%
      \begin{center}
        \begin{picture}(12,13)(-7,-16)
          \put(-3,-3.5){\makebox(0,0){$\cls{\leftm p_1, p_2\rightm}{1}{\GEQ{0.2}}$}}
          \put(3,-3.5){\makebox(0,0){$\cls{\leftm \neg p_2,p_1\rightm}{1}{\GEQ{0.1}}$}}
          %
          % In the following, we see the very first occurrence of the \step command. Its effect is to `hide' its
          % argument on the first slide of the sequence generated by \stepwise and display it on all other slides of
          % this sequence. The second \step command starts displaying on the third slide and so on...
          %
          \step{%
          % We use \afterstep{\pageTransitionDissolve} to make this first step of the diagram appear with a fancy page
          % transition. We have to use \afterstep because the page transition setting would be undone by the group
          % closing contained in \end{picture}.
            \afterstep{\pageTransitionDissolve}%
            \put(-3,-4){\line(3,-1){3}}
            \put(3,-4){\line(-3,-1){3}}
            \put(0,-4.5){\makebox(0,0){(ass.)}}
            \put(0,-5.5){%
              \makebox(0,0){%
              %
              % Here, one use of \step is nested inside the other. 
              %
              % Usually `hiding' means ignoring altogehter. For the following application however, the `hidden' objects
              % are nested inside a brace. This means that instead of ignoring its argument, it's better if
              % \step displays an appropriate amount of blank space. This behaviour (create blank space) is exhibited by
              % the custom command \bstep (for `boxed' step).
              %
              % The command \rebstep allows both boxes to appear simultaneously.
              %
              % \afterstep is used again to reset the page transition to Replace for the building of the formula.
              % 
              % Setting \activatestep to \highlightboxed makes the additional formulae `stick out' when they are filled
              % in. 
              % 
                \let\activatestep=\highlightboxed
                $
                \cls
                {\leftm p_1, \bstep{\afterstep{\pageTransitionReplace}p_1}, p_2,\rebstep{\neg p_2}\rightm}
                {2.2}
                {\rebstep{\GEQ{0.1}}}%
                $%
                }%
              }
            }
          \step{%
            \afterstep{\pageTransitionDissolve}%
            \put(0,-6){\line(0,-1){1}}
            \put(0.2,-6.5){\makebox(0,0)[l]{(removing)}}
            \put(0,-7.5){\makebox(0,0){$\cls{\leftm p_1, p_1\rightm}{1.2}{\GEQ{0.1}}$}}
            }
          \step{%
            \put(-8,-7.5){\makebox(0,0){$\left[\neg p_1,\GEQ{0.4}\right]$}}
            \put(-8,-8){\line(5,-1){5}}
            \put(0,-8){\line(-3,-1){3}}
            \put(-3,-8.5){\makebox(0,0){(ass.)}}
            \put(-3,-9.5){\makebox(0,0){$\cls{\leftm p_1, p_1,\neg p_1\rightm}{1.6}{\GEQ{0.1}}$}}
            \put(-3,-10){\line(0,-1){1}}
            \put(-2.8,-10.5){\makebox(0,0)[l]{(removing)}}
            \put(-3,-11.5){\makebox(0,0){$\cls{\leftm p_1\rightm}{0.6}{\GEQ{0.1}}$}}
            }
          \step{%
            \put(-8,-8){\line(0,-1){4}}
            \put(-8,-12){\line(5,-1){5}}
            \put(-3,-12){\line(0,-1){1}}
            \put(-2.8,-12.5){\makebox(0,0)[l]{(assembling)}}
            \put(-3,-13.5){\makebox(0,0){$\cls{\leftm p_1,\neg p_1\rightm}{1.0}{\GEQ{0.1}}$}}
            \put(-3,-14){\line(0,-1){1}}
            \put(-2.8,-14.5){\makebox(0,0)[l]{(removing)}}
            \put(-3,-15.5){\makebox(0,0){$\left[\fbox{\phantom{x}},\GEQ{0.1}\right]$}}
            }
        \end{picture}%
      \end{center}
      }%
    }

  % The whole execution of \stepwise is encapsuled in a group to make all changes local. This affects also the setting
  % of the page transition to dissolve in the last but one step. We set it again explicitly to make the last step appear
  % with this page transition as well.

  \pageTransitionDissolve


%%% Local Variables: 
%%% mode: latex
%%% fill-column: 120
%%% TeX-master: "picexample"
%%% End: 
}
  

%%% Local Variables: 
%%% mode: latex
%%% TeX-master: "picexample"
%%% End: 

\end{slide}
\end{document}



% Local Variables: 
% fill-column: 120
% TeX-master: t
% End: 
