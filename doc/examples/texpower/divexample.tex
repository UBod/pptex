%-----------------------------------------------------------------------------------------------------------------
% File: divexample.tex
%
% Divisibility example (demonstrating \step's optional arguments) for the package texpower.sty.
% 
% This file can be compiled with pdfLaTeX or (standard) LaTeX. When using standard LaTeX, the dvi file produced should
% be processed with 
% 
% dvips -Ppdf -j0 divexample
% 
% afterwards processing the resulting ps file with
%
% distill divexample.ps
%
% (The syntax is for a unix system with tetex 1.0 and distiller 3. Modify appropriately for other configurations.)
%
% The resulting pdf file is meant for presenting `interactively' with Adobe Acrobat Reader. 
%
%-----------------------------------------------------------------------------------------------------------------
% Autor: Stephan Lehmke <Stephan.Lehmke@cs.uni-dortmund.de>
%
% v0.0.1 Mar 20, 2000: First version for the pre-alpha release of TeXPower.
%

%-----------------------------------------------------------------------------------------------------------------
% We input the generic preamble.

%-----------------------------------------------------------------------------------------------------------------
% File: __TPpble.tex
%
% Code for the preamble of TeXPower demos.
% 
% This file is input by others. Don't compile it separately.
%
%-----------------------------------------------------------------------------------------------------------------
% Autor: Stephan Lehmke <Stephan.Lehmke@cs.uni-dortmund.de>
%
% v0.0.1 Mar 20, 2000: First version for the pre-alpha release of TeXPower.
%

%-----------------------------------------------------------------------------------------------------------------
% File: __TPpbla.tex
%
% Code for the first part of the preamble of TeXPower demos.
% 
% This file is input by others. Don't compile it separately.
%
%-----------------------------------------------------------------------------------------------------------------
% Autor: Stephan Lehmke <Stephan.Lehmke@cs.uni-dortmund.de>
%
% v0.0.1 Mar 20, 2000: First version for the pre-alpha release of TeXPower.
% v0.0.2 Mar 21, 2000: Remedying an incompatibility between LaTeX releases concerning the implementation of
%                      \@iiiparbox (Apr 11: this code is now part of texpower.sty). 
% v0.0.3 Apr 11, 2000: Color emphasis code moved into texpower.
%

\documentclass
[%
%-----------------------------------------------------------------------------------------------------------------
% Document class options:
% -----------------------
%
% Landscape slides formatted for letter paper fit most screen resolutions (more or less).
%
  letterpaper,%
  landscape,%
% 
% The KOMA option makes powersem load scrartcl.cls instead of article.cls.
%
  KOMA,%
% KOMA document class options are accepted.
  smallheadings,%
%
% The calcdimensions option makes powersem calculate the slide dimensions automatically from paper size and margins.
  calcdimensions,%
%
% The display option sets everything up for producing slides to be displayed interactively.
% This option is also recognized by the texpower package.
%
  display%
%-----------------------------------------------------------------------------------------------------------------
]
%-----------------------------------------------------------------------------------------------------------------
% Document class powersem, based on seminar.cls for simulating ppower via latex+distiller (instead of pdflatex).
%
{powersem}
%-----------------------------------------------------------------------------------------------------------------


%-----------------------------------------------------------------------------------------------------------------
% Set slide margins rather small for maximum use of space. This is a demo, remember.
%
\renewcommand{\slidetopmargin}{5mm}
\renewcommand{\slidebottommargin}{5mm}

\renewcommand{\slideleftmargin}{5mm}
\renewcommand{\sliderightmargin}{5mm}


%-----------------------------------------------------------------------------------------------------------------
% We need some more packages...
%

\usepackage{url}

\usepackage[latin1]{inputenc}

% One more Text emphasis command...

\let\name=\textsc


%%% Local Variables: 
%%% mode: latex
%%% fill-column: 120
%%% TeX-master: nil
%%% End: 

%-----------------------------------------------------------------------------------------------------------------
% File: __TPpblb.tex
%
% Code for the second part of the preamble of TeXPower demos.
% 
% This file is input by others. Don't compile it separately.
%
%-----------------------------------------------------------------------------------------------------------------
% Autor: Stephan Lehmke <Stephan.Lehmke@cs.uni-dortmund.de>
%
% v0.0.1 Mar 20, 2000: First version for the pre-alpha release of TeXPower.
% v0.0.2 Mar 22, 2000: Now loading the config file.
% v0.0.3 Mar 29, 2000: texpower doesn't load hyperref any more; there's now a package fixseminar. 
% v0.0.4 Apr 19, 2000: Added \slidetitle command. 
%


%-----------------------------------------------------------------------------------------------------------------
% We load hyperref and fixseminar which fixes some problems with seminar.
%
\usepackage[bookmarksopen,colorlinks,urlcolor=red,pdfpagemode=FullScreen]{hyperref}
\usepackage{fixseminar}

%-----------------------------------------------------------------------------------------------------------------
% Finally, the texpower package is loaded. 
%
\usepackage{texpower}

%% The configuration file allows user-specific settings.

%-----------------------------------------------------------------------------------------------------------------
% File: __TP.cfg
%
% Code for user-specific configuration of TeXPower documentation files.
% 
% This file is input by others. Don't compile it separately.
%
%-----------------------------------------------------------------------------------------------------------------
% Autor: Stephan Lehmke <Stephan.Lehmke@cs.uni-dortmund.de>
%
% v0.0.1 Mar 22, 2000: First version for the pre-alpha release of TeXPower.
%

\hypersetup{baseurl={http://ls1-www.cs.uni-dortmund.de/\string~lehmke/texpower/doc/}}

\hypersetup{pdfsubject={Documentation and Examples for the texpower package}}

\hypersetup{pdfauthor={Stephan Lehmke}}


%%% Local Variables: 
%%% mode: latex
%%% TeX-master: nil
%%% End: 


%-----------------------------------------------------------------------------------------------------------------
% Some more parameters...
%
\slidesmag{5}
\slideframe{none}
\pagestyle{empty}
\setcounter{tocdepth}{2}

%-----------------------------------------------------------------------------------------------------------------
% The following command produces a title page for every example and documentation file.

\newcommand{\makeslidetitle}[1]
{%
  \title{The \TeX Power bundle\\[2ex]{\normalfont #1}}
  \author{Stephan Lehmke\\\url{mailto:Stephan.Lehmke@cs.uni-dortmund.de}}
  \maketitle

  \newslide
  \setcounter{firststep}{1}% This way, the first step of all examples is displayed.
}

%%% Local Variables: 
%%% mode: latex
%%% fill-column: 120
%%% TeX-master: "dummy"
%%% End: 


%%% Local Variables: 
%%% mode: latex
%%% TeX-master: nil
%%% End: 

\hypersetup{pdftitle={texpower divisibility example}}


%-----------------------------------------------------------------------------------------------------------------
% Finally, everything is set up. Here we go...
%
\begin{document}
\begin{slide}
  % As the sectioning in the example files starts with \subsection, we `grade down' the sectioning commands. 
  \let\subsubsection\subsection
  \let\subsection\section
  %-----------------------------------------------------------------------------------------------------------------
% File: __TPdiv.tex
%
% Code for the divisibility example (demonstrating \step's optional arguments) for the package texpower.sty.
% 
% This file is input by others. Don't compile it separately.
%
%-----------------------------------------------------------------------------------------------------------------
% Autor: Stephan Lehmke <Stephan.Lehmke@cs.uni-dortmund.de>
%
% v0.0.1 Mar 20, 2000: First version for the pre-alpha release of TeXPower.
%
% v0.0.2 Apr 26, 2000: Some small changes in preparation of the update to TeXpower v0.0.7.
%
% v0.0.3 May 18, 2000: New file name.
%

%-----------------------------------------------------------------------------------------------------------------
%
\makeslidetitle{\macroname{stepwise} Example: Fooling Around}\label{Sec:ExFooling}
% The following example is just to show that a lot of fancy things are possible by appropriately defining the diverse
% `hooks' offererd by \stepwise.
`Tweaking' the hooks a little allows some truly bizarre applications\dots

% In the following \stepwise command, \activatestep will be set to \textbf. This means the `first' occurrence of a number
% is wider (because bf is an extended font) than all other occurrences of it. To avoid glitches in line breaks, all
% other occurrences of numbers (visible and invisible) need to be of the same width. Hence we define new versions of
% \displaystepcontents and \hidestepcontents.
\def\mydisplay#1{\makebox[\widthof{\textbf{#1}}]{#1}}
\def\myhide#1{\phantom{\makebox[\widthof{\textbf{#1}}]{#1}}}

\newcounter{modulo}
\newcounter{i}

% Finally, we define a custom \step command which sets the optional arguments of \step. 
% We already have introduced the first optional argument, which determines when a \step is activated for the first
% time. Here, this one is left at its default value (\value{step}=\value{stepcommand}). 
% Here, we also use the second optional argument, which has the same syntax as the first one (i.e. it can be surrounded
% either by square brackets or braces), but determines when a \step is active _at all_ (i.e. whether
% \displaystepcontents or \hidestepcontents is used). The default is the internal check \if@first@TP@true, which determines
% whether step number \value{stepcommand} has already been activated for the first time (this is saved internally for
% every step). Now, we redefine this condition to be fulfilled whenever the value of the counter step is divisible by
% the value of the counter stepcommand. 
\def\mystep{\step[](\setcounter{modulo}{\value{step}/\value{stepcommand}*\value{stepcommand}}\ifthenelse{\value{step}=\value{modulo}})}%


% By setting the page duration to 0.5 seconds, we make the following sequence of slides appear automatically one by one,
% one every half second.

\pageDuration{0.5}

% 
% We use the custom command \parstepwise which not only wraps the whole argument of \stepwise into a minipage (because
% otherwise vertical spacing goes haywire, don't ask me why), but also gives substance to steps. We use the starred
% version of this command so that the number 1 appears on the first slide generated by \parstepwise.
%
% \activatestep is set to \textbf to emphasize the number the divisors of which are displayed.
\parstepwise*[\let\activatestep=\textbf\let\displaystepcontents=\mydisplay\let\hidestepcontents=\myhide]
{%
  \huge
  \setcounter{i}{0}%
    Divisibility demo:
    
    % We just have to generate the appropriate number of \mystep commands which display `their' numbers.
    \whiledo{\value{i}<40}{\stepcounter{i}\mystep{\arabic{i}} }%

  }

% Stop automatic advancing of pages.

\stopAdvancing

%%% Local Variables: 
%%% mode: latex
%%% fill-column: 120
%%% TeX-master: "divexample"
%%% End: 

\end{slide}
\end{document}



% Local Variables: 
% fill-column: 120
% TeX-master: t
% End: 
