%-----------------------------------------------------------------------------------------------------------------
% File: __TPpblb.tex
%
% Code for the second part of the preamble of TeXPower demos.
% 
% This file is input by others. Don't compile it separately.
%
%-----------------------------------------------------------------------------------------------------------------
% Autor: Stephan Lehmke <Stephan.Lehmke@cs.uni-dortmund.de>
%
% v0.0.1 Mar 20, 2000: First version for the pre-alpha release of TeXPower.
% v0.0.2 Mar 22, 2000: Now loading the config file.
% v0.0.3 Mar 29, 2000: texpower doesn't load hyperref any more; there's now a package fixseminar. 
% v0.0.4 Apr 19, 2000: Added \slidetitle command. 
%


%-----------------------------------------------------------------------------------------------------------------
% We load hyperref and fixseminar which fixes some problems with seminar.
%
\usepackage[bookmarksopen,colorlinks,urlcolor=red,pdfpagemode=FullScreen]{hyperref}
\usepackage{fixseminar}

%-----------------------------------------------------------------------------------------------------------------
% Finally, the texpower package is loaded. 
%
\usepackage{texpower}

%% The configuration file allows user-specific settings.

%-----------------------------------------------------------------------------------------------------------------
% File: __TP.cfg
%
% Code for user-specific configuration of TeXPower documentation files.
% 
% This file is input by others. Don't compile it separately.
%
%-----------------------------------------------------------------------------------------------------------------
% Autor: Stephan Lehmke <Stephan.Lehmke@cs.uni-dortmund.de>
%
% v0.0.1 Mar 22, 2000: First version for the pre-alpha release of TeXPower.
%

\hypersetup{baseurl={http://ls1-www.cs.uni-dortmund.de/\string~lehmke/texpower/doc/}}

\hypersetup{pdfsubject={Documentation and Examples for the texpower package}}

\hypersetup{pdfauthor={Stephan Lehmke}}


%%% Local Variables: 
%%% mode: latex
%%% TeX-master: nil
%%% End: 


%-----------------------------------------------------------------------------------------------------------------
% Some more parameters...
%
\slidesmag{5}
\slideframe{none}
\pagestyle{empty}
\setcounter{tocdepth}{2}

%-----------------------------------------------------------------------------------------------------------------
% The following command produces a title page for every example and documentation file.

\newcommand{\makeslidetitle}[1]
{%
  \title{The \TeX Power bundle\\[2ex]{\normalfont #1}}
  \author{Stephan Lehmke\\\url{mailto:Stephan.Lehmke@cs.uni-dortmund.de}}
  \maketitle

  \newslide
  \setcounter{firststep}{1}% This way, the first step of all examples is displayed.
}

%%% Local Variables: 
%%% mode: latex
%%% fill-column: 120
%%% TeX-master: "dummy"
%%% End: 
